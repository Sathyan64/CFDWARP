\documentclass{warpdoc}
\usepackage{times}         % can use utopia, or palatino
\usepackage{mathptmx}       % math font is times
\DeclareMathSizes{10}{10}{6.8}{6.0}   % redefine the math font sizes for 10pt document
\DeclareMathSizes{11}{11}{7.48}{6.6}  % redefine the math font sizes for 11pt document
\DeclareMathSizes{12}{12}{8.16}{7.2}  % redefine the math font sizes for 12pt document
\newlength\lengthfigure                  % declare a figure width unit
\setlength\lengthfigure{0.158\textwidth} % make the figure width unit scale with the textwidth
\usepackage{psfrag}         % use it to substitute a string in a eps figure
\usepackage{subfigure}
\usepackage{rotating}
\usepackage{pstricks}
\usepackage[innercaption]{sidecap} % the cute space-saving side captions
\usepackage{scalefnt}

%%%%%%%%%%%%%=--NEW COMMANDS BEGINS--=%%%%%%%%%%%%%%%%%%%%%%%%%%%%%%%%%%
\newcommand{\alb}{\vspace{0.2cm}\\} % array line break
\newcommand{\mfd}{\displaystyle}
\renewcommand{\fontsizetable}{\footnotesize\scalefont{0.7}}
\setcounter{tocdepth}{3}
\let\citen\cite

%%%%%%%%%%%%%=--NEW COMMANDS ENDS--=%%%%%%%%%%%%%%%%%%%%%%%%%%%%%%%%%%%%
%%%%%%%%%%%%%=--NEW COMMANDS BEGINS--=%%%%%%%%%%%%%%%%%%%%%%%%%%%%%%%%%%

\newcommand{\ns}{{{n}_{\rm s}}}
\newcommand{\efficiency}{\eta}
\newcommand{\ordi}{{\rm d}}
\newcommand{\unitvecdiff}[2]{\overline{\vec{#1} - \vec{#2}}}
%\let\vec\bf
%\renewcommand{\vec}[1]{\pmb{#1}}
\newcommand{\betae}{\beta_{\rm e}}
\newcommand{\betai}{\beta_{\rm i}}
\newcommand{\sigmatilde}{\widetilde{\sigma}}
\newcommand{\mdot}{\dot{m}}
\newcommand{\efp}{\psi}
\newcommand{\vion}{\vec{v^{\rm i}}}
\newcommand{\vneutral}{\vec{v^{\rm n}}}
\newcommand{\velectron}{\vec{v^{\rm e}}}
\newcommand{\vflow}{\vec{v}}
\newcommand{\betaiprime}{\betai^\prime}
\newcommand{\betaeprime}{\betae^\prime}
\newcommand{\Eprime}{\vec{E}^\prime}
\newcommand{\tauei}{\tau_{\rm ei}}
\newcommand{\tauen}{\tau_{\rm en}}
\newcommand{\tauin}{\tau_{\rm in}}
\newcommand{\momin}{{k_{\rm in}}}
\newcommand{\momen}{{k_{\rm en}}}
\newcommand{\momei}{{k_{\rm ei}}}
\newcommand{\ie}{{\it i.e.}}
\newcommand{\rhoeDveDt}{\rho_{\rm e} \frac{{\rm D}v_{\rm e}}{{\rm D} t}}
\newcommand{\rhoiDviDt}{\rho_{\rm i} \frac{{\rm D}v_{\rm i}}{{\rm D} t}}
\newcommand{\gradPe}{\nabla P_{\rm e}}
\newcommand{\gradPi}{\nabla P_{\rm i}}
\newcommand{\zetae}{\zeta_{\rm e}}
\newcommand{\zetai}{\zeta_{\rm i}}
\newcommand{\zetan}{\zeta_{\rm n}}
\newcommand{\ev}{{e_{\rm v}}}
\newcommand{\ee}{{e_{\rm e}}}
\newcommand{\evzero}{{e_{\rm v}^0}}
\newcommand{\eezero}{{e_{\rm e}^0}}

\author{
  Bernard Parent
}

\email{
  bparent@princeton.edu
}

\department{
  Department of Mechanical and Aerospace Engineering
}

\institution{
  Princeton University
}

\title{
 15 species Low Temperature Air Chemical Model
}

\date{
  January -- February 2005
}

%\setlength\nomenclaturelabelwidth{0.13\hsize}  % optional, default is 0.03\hsize
%\setlength\nomenclaturecolumnsep{0.09\hsize}  % optional, default is 0.06\hsize

\nomenclature{

  \begin{nomenclaturelist}{Roman symbols}
   \item[$a$] speed of sound
  \end{nomenclaturelist}
}


\abstract{
abstract
}

\begin{document}
  \pagestyle{headings}
  \pagenumbering{arabic}
  \setcounter{page}{1}
%%  \maketitle
  \makewarpdoctitle
%  \makeabstract
%  \tableofcontents
%  \makenomenclature
%%  \listoftables
%%  \listoffigures



  







\section{Reaction Rates}










%
\begin{table*}
  \center\fontsizetable
  \begin{threeparttable}
    \tablecaption{15-species 76-reactions low temperature air model valid in the temperature range 300-600 K. The species consist of $\rm e^-$, $\rm O$, $\rm O^-$, $\rm O^+$, $\rm O_2$, $\rm O_2^-$, $\rm O_2^+$, $\rm O_3$, $\rm N$, $\rm N^+$, $\rm N_2$, $\rm N_2^+$, $\rm N_2(A)$, $\rm NO$, $\rm NO^+$ where $\rm N_2(A)$ stands for $\rm N_2(A^3 \Sigma_g^+)$.}
    \label{tab:reactions}
    \fontsizetable
    \begin{tabular}{llll}
    \toprule
    No.&Reaction & Rate Coefficient [cm$^3$/s or cm$^6$/s] & Reference \\
    \midrule 
    1a  & $\rm e^- + N_2   \rightarrow N_2^+ + e^- + e^-$  
       &  $10^{-8.3 -36.5/\vartheta}$ cm$^3$/s~~ for $3\le\vartheta\le 30$
       & Ref.\ \citen{misc:1992:kossyi}\\
    1b  & $\rm e^- + O_2   \rightarrow O_2^+ + e^- + e^-$  
       &  $10^{-8.8 -28.1/\vartheta}$ cm$^3$/s~~ for $3\le\vartheta\le 30$
       & Ref.\ \citen{misc:1992:kossyi}\\
    2A &
    \begin{tabular}{l}
      $\!\!\!\!\!\!\!\!\rm e^{-}+e^{-}+A^+ \rightarrow e^- + A$ \\
      for $\rm A \rightarrow ~N_2,~O_2,~NO,~N,~O$\\
    \end{tabular} 
       & $10^{-19} (300/T_{\rm e})^{4.5}$ cm$^6$/s
       & Ref.\ \citen{misc:1992:kossyi}\\
    3a & $\rm e^-+O_2^+ \rightarrow O + O$  
       & $2.0 \times 10^{-7} (300/T_{\rm e})^{0.7}  $ cm$^3$/s
       & Ref.\ \citen{misc:1997:aleksandrov}\\
    3b & $\rm e^-+N_2^+ \rightarrow N + N$  
       & $2.8 \times 10^{-7} (300/T_{\rm e})^{0.5}  $ cm$^3$/s 
       & Ref.\ \citen{misc:1992:kossyi}\\
    3c  & $\rm e^- + NO^+  \rightarrow N + O$  
       &  $4\times 10^{-7} \times (300/T_{\rm e})^{1.5}$  cm$^3$/s
       & Ref.\ \citen{misc:1992:kossyi}\\
    4A &
    \begin{tabular}{l}
      $\!\!\!\!\!\!\!\!\rm e^{-}+A^+ +M \rightarrow A+M$ \\
      for $\rm A \rightarrow ~N_2,~O_2,~NO,~N,~O$\\
      and $\rm M \rightarrow ~N_2,~O_2$\\
    \end{tabular} 
       & $6 \times 10^{-27} (300/T_{\rm e})^{1.5}$ cm$^6$/s
       & Ref.\ \citen{misc:1992:kossyi}\\
    5a  & $\rm e^- + O_2  \rightarrow O^- + O$  
       &  $
          \left\{
          \begin{tabular}{l}
            $\!\!\!\!\!\!\! 10^{-9.3 -12.3/\vartheta}$ cm$^3$/s for $\vartheta<8$ \\
            $\!\!\!\!\!\!\! 10^{-10.2 -5.7/\vartheta}$ cm$^3$/s for $\vartheta>8$ 
	  \end{tabular} \right. $
       & Ref.\ \citen{misc:1992:kossyi}\\
    5b & $\rm e^- + O_2 +O_2 \rightarrow O_2^- + O_2$  
       & $1.4 \times 10^{-29} \frac{300}{T_{\rm e}} \exp \left( \frac{-600}{T}\right)
         \exp \left( \frac{700(T_{\rm e}-T)}{T_{\rm e} T}  \right)$ cm$^6$/s
       & Ref.\ \citen{misc:1992:kossyi}\\
    5c & $\rm e^- + O_2 + N_2 \rightarrow O_2^- + N_2$  
       & $1.07 \times 10^{-31} \left( \frac{300}{T_{\rm e}} \right)^2 \exp \left( \frac{-70}{T}\right)
         \exp \left( \frac{1500(T_{\rm e}-T)}{T_{\rm e} T}  \right)$ cm$^6$/s
       & Ref.\ \citen{misc:1992:kossyi}\\
    5d & $\rm e^- + O_3 \rightarrow O_2^- + O$  
       & $10^{-9}$ cm$^3$/s
       & Ref.\ \citen{misc:1992:kossyi}\\
    5e & $\rm e^- + O_3 \rightarrow O^- + O_2$  
       & $10^{-11}$ cm$^3$/s
       & Ref.\ \citen{misc:1992:kossyi}\\
    6a  & $\rm O_2^- + O_2 \rightarrow e^- + O_2 + O_2$  
       & $8.6 \times 10^{-10} \exp \left( \frac{-6030}{T}\right)
               \left[1-\exp \left( \frac{-1570}{T} \right)  \right]$ cm$^3$/s
       & Ref.\ \citen{book:1997:bazelyan}, Ch.\ 2\\
    6b  & $\rm O_2^- + O \rightarrow O_3 + e^-$  
       & $1.5 \times 10^{-10} $ cm$^3$/s 
       & Ref.\ \citen{misc:1992:kossyi}\\
    6c  & $\rm O_2^- + N_2 \rightarrow O_2 + N_2 + e^-$  
       & $1.9 \times 10^{-12} \times (T/300)^{0.5} \exp(-4990/T) $ cm$^3$/s 
       & Ref.\ \citen{misc:1992:kossyi}\\
    6d  & $\rm O_2^- + O_2 \rightarrow O_2 + O_2 + e^-$  
       & $2.7 \times 10^{-10} \times (T/300)^{0.5} \exp(-5590/T) $ cm$^3$/s 
       & Ref.\ \citen{misc:1992:kossyi}\\
    7a  & $\rm O^- + O \rightarrow O_2 + e^-$  
       & $5 \times 10^{-10} $ cm$^3$/s 
       & Ref.\ \citen{misc:1992:kossyi}\\
    7b  & $\rm O^- + N \rightarrow NO + e^-$  
       & $2.6 \times 10^{-10} $ cm$^3$/s 
       & Ref.\ \citen{misc:1992:kossyi}\\
    7c  & $\rm O^- + O_2 \rightarrow O_3 + e^-$  
       & $5 \times 10^{-15} $ cm$^3$/s 
       & Ref.\ \citen{misc:1992:kossyi}\\
    7d  & $\rm O^- + N_2(A) \rightarrow O+N_2 + e^-$  
       & $2.2 \times 10^{-9} $ cm$^3$/s 
       & Ref.\ \citen{misc:1992:kossyi}\\
    
    8A &
    \begin{tabular}{l}
      $\!\!\!\!\!\!\!\!\rm A^{-}+B^{+} \rightarrow A + B$ \\
      for $\rm A^{-} \rightarrow ~O_2^-,~O^-,~O_3^-,~NO^{-},~NO_2^-~NO_3^-,~N_2O^-$\\
      and $\rm B^{+} \rightarrow ~N_2^+,~O_2^+,~N^+,~O^+,~NO^+,~NO_2^+,~N_2O^+$\\
    \end{tabular} 
       & $2.0 \times 10^{-7} (300/T)^{0.5}$ cm$^3$/s
       & Ref.\ \citen{misc:1992:kossyi}\\
    8B &
    \begin{tabular}{l}
      $\!\!\!\!\!\!\!\!\rm A^{-}+B^{+} + M\rightarrow A + B +M$ \\
      for $\rm A \rightarrow ~O_2,~O$\\
      and $\rm B \rightarrow ~N_2,~O_2,~N,~O,~NO$\\
      and $\rm M \rightarrow ~N_2,~O_2$\\
    \end{tabular} 
       & $2.0 \times 10^{-25} (300/T)^{2.5}$ cm$^6$/s  
       & Ref.\ \citen{misc:1992:kossyi}\\
    9a  & $\rm O_2^+ + N \rightarrow NO^+ + O$  
       & $1.2 \times 10^{-10} $ cm$^3$/s 
       & Ref.\ \citen{misc:1992:kossyi}\\
    9b  & $\rm N_2^+ + O \rightarrow NO^+ + N$  
       & $1.3 \times 10^{-10} (300/T)^{0.5} $ cm$^3$/s 
       & Ref.\ \citen{misc:1992:kossyi}\\
    
    10a  & $\rm N_2 + e^- \rightarrow N_2(A) + e^-$  
       & $10^{-8.4 - 14/\vartheta}$ cm$^3$/s
       & Ref.\ \citen{misc:1992:kossyi}\\
    10b  & $\rm N_2(A) + O_2 \rightarrow N_2 + O +O$  
       & $2.54 \times 10^{-12}$ cm$^3$/s
       & Ref.\ \citen{misc:1992:kossyi}\\
    11a  & $\rm N + O+N_2 \rightarrow NO + N_2$  
       & $1.76 \times 10^{-31} T^{-0.5}$ cm$^6$/s
       & Ref.\ \citen{misc:1992:kossyi}\\
    11b  & $\rm N + O+O_2 \rightarrow NO + O_2$  
       & $1.76 \times 10^{-31} T^{-0.5}$ cm$^6$/s
       & Ref.\ \citen{misc:1992:kossyi}\\
    11c  & $\rm O + O_2+N_2 \rightarrow O_3 + N_2$  
       & $6.2 \times 10^{-34} (300/T)^{2}$ cm$^6$/s
       & Ref.\ \citen{misc:1992:kossyi}\\
    11d  & $\rm O + O_2+O_2 \rightarrow O_3 + O_2$  
       & $6.9 \times 10^{-34} (300/T)^{1.25}$ cm$^6$/s
       & Ref.\ \citen{misc:1992:kossyi}\\
    11E &
      $\rm N+N + M\rightarrow N_2 + M$ for $\rm M \rightarrow ~N_2,~O_2$
       & $8.27 \times 10^{-34} \exp (500/T)$ cm$^6$/s ??  
       & Ref.\ \citen{misc:1992:kossyi}\\
    11f  & $\rm O + O +O_2 \rightarrow 2 O_2 $  
       & $2.45 \times 10^{-31} T^{-0.63} $ cm$^6$/s  ?? 
       & Ref.\ \citen{misc:1992:kossyi}\\
    11g  & $\rm O + O +N_2 \rightarrow O_2 + N_2$  
       & $2.76 \times 10^{-34} \exp (720/T) $ cm$^6$/s ??
       & Ref.\ \citen{misc:1992:kossyi}\\
    11h  & $\rm N + O_2 \rightarrow NO + O$  
       &  $
          \left\{
          \begin{tabular}{l}
            $\!\!\!\!\!\!\! 4.5 \times 10^{-12} \exp(-3220/T) $ cm$^3$/s for $200 {\rm K} \le T \le 300 {\rm K}$\\
            $\!\!\!\!\!\!\! 1.1 \times 10^{-14} T \exp(-3150/T)$ cm$^3$/s for $T>300$ K 
	  \end{tabular} \right. $
       & Ref.\ \citen{misc:1992:kossyi}\\
    
    
    \midrule    
    
    4f  & $\rm e^- + O_3  \rightarrow O_2 + O + e^-$  
       &  $10 \times (10^{-7.9-13.4/\vartheta}+10^{-8 -16.9/\vartheta}+10^{- 8.8 -11.9/\vartheta})$ cm$^3$/s
       & Ref.\ \citen{misc:1992:kossyi}\\
    
    7d  & $\rm NO + e^- \rightarrow NO^+ + e^- + e^-$  
       &  ? cm$^3$/s
       & Ref.\ \citen{misc:1992:kossyi}\\
    4c & $\rm e^- + O + O_2 \rightarrow O^- + O_2$  
       & $10^{-31}$ cm$^6$/s
       & Ref.\ \citen{misc:1992:kossyi}\\
    4d & $\rm e^- + O + O_2 \rightarrow O + O_2^-$  
       & $10^{-31}$ cm$^6$/s
       & Ref.\ \citen{misc:1992:kossyi}\\
    4e & $\rm e^- + O_3 + O_2 \rightarrow O_3^- + O_2$  
       & ?? cm$^6$/s
       & Ref.\ \citen{misc:1992:kossyi}\\
    4h & $\rm e^- + NO_2 \rightarrow O^- + NO$  
       & $10^{-11}$ cm$^3$/s
       & Ref.\ \citen{misc:1992:kossyi}\\
    6c  & $\rm O_2^- + N \rightarrow NO_2 + e^-$  
       & $5 \times 10^{-10} $ cm$^3$/s 
       & Ref.\ \citen{misc:1992:kossyi}\\
    7d  & $\rm O^- + NO \rightarrow NO_2 + e^-$  
       & $2.6 \times 10^{-10} $ cm$^3$/s 
       & Ref.\ \citen{misc:1992:kossyi}\\
    10b  & $\rm N_2(A) + O_2 \rightarrow N_2 O + O$  
       & $7.8 \times 10^{-14}$ cm$^3$/s
       & Ref.\ \citen{misc:1992:kossyi}\\
    
    \bottomrule
    \end{tabular}
   \end{threeparttable}
\end{table*}
%

Two good sources of reaction rates: Kossyi \cite{misc:1992:kossyi}
and M\"atzing \cite{misc:1991:matzing}. M\"atzing is especially important to determine the
chemical reactions that would occur due to the electron beam: we'll check this out later.
Another paper listing chemical reactions of ionized air is Ref.\ \citen{misc:1997:aleksandrov} where a model is proposed to tackle air ionized with a discharge for a temperature varying between 300 K and 5000 K. Yet other papers worth looking into are Refs.\ \citen{misc:2000:bourdon} and \citen{aiaaconf:1999:laux}.


For now, we'll base our set of chemical reactions on the one outlined in Ref.\ \citen{misc:2002:macheret} which was developped to solve low temperature air (300 K to 600 K) ionized with electron beams. For simplicity, chemical reactions related to the electron beam as well as chemical reactions that would occur at higher temperature are not taken into account at this stage. The 15-species 76-reaction model is shown in Table \ref{tab:reactions}.
The species consist of $\rm O_2$, $\rm N_2$, $\rm O$, $\rm N$, $\rm O_2^+$, $\rm N_2^+$, $\rm N^+$, $\rm O^+$, $\rm O_2^-$, $\rm e^-$, $\rm NO$, $\rm O_3$, $\rm NO^+$, $\rm O^-$, $\rm N_2(A^3 \Sigma_g^+)$. The symbol $\vartheta$ stands for the reduced electric field $|\vec{E}^\star|/N$ in units of $10^{-16}$ V cm$^2$.

As outlined in Ch.\ 13 of Ref.\ \citen{book:1989:anderson}, we can write the $n$th reaction in the form:
%
\begin{equation}
  \sum_{k=1}^\ns m_{n,k}^{\rm R} X_k \rightarrow \sum_{k=1}^\ns m_{n,k}^{\rm P} X_k
\end{equation}
%
where $X_k$ refers to the name of the $k$th species, while $m_{n,k}^{\rm R}$ and $m_{n,k}^{\rm P}$
refer to integer numbers preceding the reactants and products, respectively. 
 Should the reaction rate of the $n$th reaction be denoted as ${k_f}_n$,
the rate of change of the number density $N_{k}$ due to the $n$th reaction can be expressed as:
%
\begin{equation}
  \left. \frac{{\rm d} N_{k}}{{\rm d} t} \right|_n = \left( m_{n,k}^{\rm P} - m_{n,k}^{\rm R} \right) {k_f}_n \prod_{i=1}^\ns N_{i}^{m_{n,i}^{\rm R}}
\end{equation}
%
where the number density $N_{k}$ is in  1/cm$^3$ and can be obtained from the partial density as follows:
%
\begin{equation}
  N_{k} {\rm ~~(1/cm^3)}= \frac{\rho_k {\rm ~~(kg/m^3)}~~~{\cal A ~~{\rm (1/mole)}}}{{\cal M}_k {\rm ~~(kg/mole)}} \times 10^{-6} {\rm ~~(m^3/cm^3)}
\end{equation}
%
where ${\cal M}_k$ is the molecular weight of the $k$th species. 
Then, the continuity equations source terms related to the chemical reactions can be expressed as:
%
\begin{equation}
 W_k  = \frac{10^6 \times {\cal M}_k}{\cal A} \sum_{n=1}^{n_{\rm r}} \left. \frac{{\rm d} N_{k}}{{\rm d} t} \right|_n     
\end{equation}
%
where $W_k$ is in $\rm kg/m^3 s$, ${{\rm d} n_k}/{{\rm d} t}$ in $\rm 1/cm^3 s$, and ${\cal M}_k$ in $\rm kg/mole$ and $\cal A$ is in 1/mole.

\bibliographystyle{warpdoc}
\bibliography{local}


\end{document}


