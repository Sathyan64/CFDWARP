\section{Computational Co-Ordinates Transformation}

	Having phrased the Navier-Stokes equations in matrix form (Eqs. \ref{eqn:final}-\ref{eqn:finalHv}), 
the next step is to cast these equations in a form suitable for numerical simulation.  To do this, two things
are required to make this transformation as useful as possible.  The first, and most involved
step, is to transform the equations onto computational co-ordinates.  This will allow 
for the fact that the orthogonal computational grid need not necessarily be aligned with the space 
physical co-ordinates  
of the system under consideration, and thus cannot be applied directly to an arbitrary physical gridding.  However,
by using computational co-ordinates one can easily allow for any type of grid (i.e.
clustered along given directions or in specified areas) through the incorporation of metric 
terms in the governing equations.

	The second step is to phrase the variables in a language suitable for computation, which
involves only direct substitution and is hence fairly straightforward.  However one would like to 
have indices that match those used by the programming language, as although in reality the number of
species goes from 1 to $n_s$, for a programming language such as \emph{C} the indices would vary 
from 0 to $n_s - 1$.  As well, common variables differentiated by subscripts alone rather than distinct
Greek or arithmetic symbols are easier to program, as these can be represented by arrays of various dimensions.


%---------------------------------------------------------------------------------------------------------
\subsection{Metric Derivatives}

	In order to properly transform the governing equations from the Cartesian to the computational
co-ordinate system metric derivatives will be needed, as it is these terms that describe the geometrical
relationship between the two domains.  Thus before tackling the main set of equations, it is helpful to 
derive relations for the metric terms and have them at the ready for substitution into the governing
equations when needed.   We can make use of the fact that there are only two independent space variables, 
$x$ and $r$, (in axisymmetric flow) and one variable in time, $t$, which allows a slight simplification from
a fully three dimensional transformation.  Making use of the relationship between differential quantities as 
defined by the Calculus one can write,

\begin{displaymath}
	\begin{array}{c}
		dt=\frac{\partial t}{\partial \tau}d\tau + \frac{\partial t}{\partial \xi}d\xi +
		\frac{\partial t}{\partial \eta}d\eta \\
		dx=\frac{\partial x}{\partial \tau}d\tau + \frac{\partial x}{\partial \xi}d\xi +
		\frac{\partial x}{\partial \eta}d\eta \\
		dr=\frac{\partial r}{\partial \tau}d\tau + \frac{\partial r}{\partial \xi}d\xi +
		\frac{\partial r}{\partial \eta}d\eta
	\end{array}
\end{displaymath}

	Now since real time is independent of the physical grid location (and hence independent of the computational
grid location as there is a one to one correspondence between points in the two domains) then one can write 
$\frac{\partial t}{\partial \xi}$ = 
$\frac{\partial t}{\partial \eta}$ = 0, and if one further assumes that computational time is identical to
real time then $\frac{\partial t}{\partial \tau}$ = 1, the above relations can be expressed in matrix form
as,

\begin{equation}
	\begin{array}{cccc}
		\left\{
		\begin{array}{c}
		dt \\ dx \\ dr 
		\end{array} 
		\right\}
		& = &
		\left[
		\begin{array}{ccc}
		1 & 0 & 0 \\ x_{t} & x_{\xi} & x_{\eta} \\ r_{t} & r_{\xi} & r_{\eta} 
		\end{array}
		\right] 
		&
		\left\{
		\begin{array}{c}
		d\tau \\ d\xi \\ d\eta 
		\end{array}
		\right\}
	\end{array} 
\label{eqn:matrix1}
\end{equation}

	Using the Calculus again, one can express these relations from the opposite point of view 
expressing the differential computational quantities in terms of derivatives with respect to the 
original Cartesian co-ordinates,

\begin{displaymath}
	\begin{array}{c}
		d\tau=\frac{\partial \tau}{\partial t}dt + \frac{\partial \tau}{\partial x}dx +
		\frac{\partial \tau}{\partial r}dr \\
		d\xi=\frac{\partial \xi}{\partial t}dt + \frac{\partial \xi}{\partial x}dx +
		\frac{\partial \xi}{\partial r}dr \\
		d\eta=\frac{\partial \eta}{\partial t}dt + \frac{\partial \eta}{\partial x}dx +
		\frac{\partial \eta}{\partial r}dr
	\end{array}
\end{displaymath}

	which again, after applying the same logic as before, can be simplified and rewritten as

\begin{equation}
	\begin{array}{cccc}
		\left\{
		\begin{array}{c}
		d\tau \\ d\xi \\ d\eta 
		\end{array} 
		\right\}
		& = &
		\left[
		\begin{array}{ccc}
		1 & 0 & 0 \\ \xi_t & \xi_x & \xi_r \\ \eta_t & \eta_x & \eta_r 
		\end{array}
		\right] 
		&
		\left\{
		\begin{array}{c}
		dt \\ dx \\ dr 
		\end{array}
		\right\}
	\end{array} 
\label{eqn:matrix2}
\end{equation}

	Examining Eqs. \ref{eqn:matrix1} and \ref{eqn:matrix2} it can be seen that the square matrices in 
each equation are inverses of each other, thus using the property of inverse matrices that states $AA^{-1} = I$ one can
write

\begin{equation}
\begin{array}{ccccc}
	\begin{array}{c}
		\left[
		\begin{array}{ccc}
		1 & 0 & 0 \\ x_{\tau} & x_{\xi} & x_{\eta} \\ r_{\tau} & r_{\xi} & r_{\eta} 
		\end{array}
		\right] 
	\end{array} 
& \cdot  &
	\begin{array}{c}
		\left[
		\begin{array}{ccc}
		1 & 0 & 0 \\ \xi_t & \xi_x & \xi_r \\ \eta_t & \eta_x & \eta_r 
		\end{array}
		\right] 
	\end{array}
& = &
	\begin{array}{c}
		\left[
		\begin{array}{ccc}
		1 & 0 & 0 \\ 0 & 1 & 0 \\ 0 & 0 & 1
		\end{array}
		\right]
	\end{array}
\end{array}
\label{eqn:identity} 
\end{equation}

	Eq. \ref{eqn:identity} yields nine equations when multiplied out, three of which provide no useful
relations (i.e. 0 = 0), two provide relations for the metric derivatives with respect to time (which in the case of a grid
that is constant in time reduce to zero) and the remaining four express relations for the pertinent metric derivatives

\begin{displaymath}
	\begin{array}{c}
	\xi_t = -x_t \xi_x - r_t \xi_r \\
	\eta_t = -x_t \eta_x - r_t \eta_r \\
	\xi_x = \frac{1}{x_\xi}(1 - r_\xi \xi_r) \\
	\xi_x = -\frac{1}{x_\eta}(r_\eta \xi_r) \\
	\eta_r = -\frac{1}{r_\xi}(x_\xi \eta_x) \\
	\eta_r = \frac{1}{r_\eta}(1 - x_\eta \eta_x)
	\end{array}
\end{displaymath}

	After suitable manipulations of the last four equations above (eliminating one of the metric derivatives, 
solving for the remaining metric derivative, and then substituting this result back into the equations to solve
for the eliminated variable) one arrives at the following results,

\begin{displaymath}
	\begin{array}{c}
	\xi_x = J r_\eta \\
	\xi_r = -J x_\eta \\
	\eta_x = -J r_\xi \\
	\eta_r = J x_\xi
	\end{array}
\end{displaymath}

	where is J is the Jacobian of transformation (also known as the metric Jacobian) and is defined as,

\begin{equation}
	J = \frac{1}{x_\xi r_\eta - x_\eta r_{\xi}}
\label{eqn:J}
\end{equation}

	Substituting the above results in the previous expressions for $\xi_t$ and $\eta_t$ also yields

\begin{equation}
	\xi_t = J(-x_t r_\eta + r_t x_\eta)
\label{eqn:xit}
\end{equation}

\begin{equation}
	\eta_t = J(x_t r_\xi - r_t x_\xi)
\label{eqn:etat}
\end{equation}

	Going one step further and rephrasing the above relations (except for Eqs. \ref{eqn:xit} and \ref{eqn:etat}) 
in variables more suited to a programming
language, we will let the Cartesian co-ordinates be represented as $x$ = $x_0$ and $r$ = $x_1$ while the 
computational co-ordinates as $\xi$ = $X_0$, $\eta$ = $X_1$.  Therefore,

\begin{displaymath}
	\begin{array}{c}
	\xi_x = \frac{\partial \xi}{\partial x} = \frac{\partial X_0}{\partial x_0} = X_{0,0} = 
		J \frac{\partial x_1}{\partial X_1} \\
	\xi_r = \frac{\partial \xi}{\partial r} = \frac{\partial X_0}{\partial x_1} = X_{0,1} = 
		-J \frac{\partial x_0}{\partial X_1} \\
	\eta_x = \frac{\partial \eta}{\partial x} = \frac{\partial X_1}{\partial x_0} = X_{1,0} = 
		-J \frac{\partial x_1}{\partial X_0} \\
	\eta_r = \frac{\partial \eta}{\partial r} = \frac{\partial X_1}{\partial x_1} = X_{1,1} = 
		J \frac{\partial x_0}{\partial X_0} 
	\end{array}
\end{displaymath}

	These relations can be succinctly described by a single formula 

\begin{equation}
	X_{i,j} = J(2\delta_{i,j} -1)\frac{\partial x_{j+1}}{\partial X_{i+1}}
\label{eqn:metrics}
\end{equation}

	where $\delta_{i,j}$ is the Kronecker delta which is equal to 1 if $i = j$ and 0 
otherwise (i.e. $i \neq j$).  Also note that when the index goes higher than the number of dimensions
minus one, it cycles back to its lowest value (i.e. for two dimensions $i$ = 0,1 thus $i+1$ = 1,0 and
similarly for $j$).

%--------------------------------------------------------------------------------------------------------------
\subsection{Governing Equations} 

	Starting with the axisymmetric, multi-species, viscous, turbulent Navier-Stokes equations as
shown in Eqs. \ref{eqn:final}-\ref{eqn:finalHv}, it is desired to transform these 
equations onto the general computational co-ordinates $\xi$ and $\eta$, as well as the computational time
$\tau$.  In the following derivations, the Euler equations will be considered (i.e. the viscous terms will be
neglected) however the analysis applies equally as well to these terms and hence the results will be simply be 
extended to include the full set of equations.  Thus the axisymmetric Euler equations can be written as

\begin{equation}
	\frac{\partial Q}{\partial t} + \frac{\partial E}{\partial x} + \frac{\partial F}{\partial r} + S_{be3D} = 0
\label{eqn:euler}
\end{equation}

	Now if we divide each term in Eq. \ref{eqn:euler} by the metric Jacobian $J$ (as defined by Eq. \ref{eqn:J}) 
one can write

\begin{displaymath}
	\frac{1}{J}\frac{\partial Q}{\partial t} + \frac{1}{J}\frac{\partial E}{\partial x} 
	+ \frac{1}{J}\frac{\partial F}{\partial r} + \frac{1}{J}S_{be3D} = 0
\end{displaymath}

	Taking a closer look at the first term, the conservative variables vector, one can express the 
derivative with respect to time using the Calculus as

\begin{displaymath}
	\frac{1}{J}\frac{\partial Q}{\partial t} = \frac{1}{J}\Big[\frac{\partial \tau}{\partial t}
	\frac{\partial Q}{\partial \tau} + \frac{\partial \xi}{\partial t}
	\frac{\partial Q}{\partial \xi} + \frac{\partial \eta}{\partial t} \frac{\partial Q}{\partial \eta}\Big]
\end{displaymath}

	Now if one adds groups of terms which in and of themselves sum to zero (these terms are shown in curly braces $\{\}$) 
one can write for the conservative variables vector 

\begin{displaymath}
	\frac{\tau_t}{J}\frac{\partial Q}{\partial \tau} + 
	\frac{\xi_t}{J} \frac{\partial Q}{\partial \xi} + \frac{\eta_t}{J} \frac{\partial Q}{\partial \eta}
	+ \{ Q\frac{\partial}{\partial \tau}(\frac{\tau_t}{J}) - Q\frac{\partial}{\partial \tau}(\frac{\tau_t}{J}) \}
	+ \{ Q\frac{\partial}{\partial \xi}(\frac{\xi_t}{J}) - Q\frac{\partial}{\partial \xi}(\frac{\xi_t}{J}) \}
	+ \{ Q\frac{\partial}{\partial \eta}(\frac{\eta_t}{J}) - Q\frac{\partial}{\partial \eta}(\frac{\eta_t}{J}) \}
\end{displaymath}

	which can be regrouped as

\begin{equation}
	\Big[\frac{\tau_t}{J}\frac{\partial Q}{\partial \tau} + Q\frac{\partial}{\partial \tau}(\frac{\tau_t}{J})\Big] +
	\Big[\frac{\xi_t}{J}\frac{\partial Q}{\partial \xi} + Q\frac{\partial}{\partial \xi}(\frac{\xi_t}{J})\Big] + 
	\Big[\frac{\eta_t}{J}\frac{\partial Q}{\partial \eta} + Q\frac{\partial}{\partial \eta}(\frac{\eta_t}{J})\Big]
	-Q \Big\{\frac{\partial}{\partial \tau}(\frac{\tau_t}{J}) + \frac{\partial}{\partial \xi}(\frac{\xi_t}{J})
	+ \frac{\partial}{\partial \eta}(\frac{\eta_t}{J}) \Big\}
\label{eqn:Qbig}
\end{equation}

	In this form it can be seen that the terms in the square brackets can be combined into a single derivative
using the inverse of the chain rule from the Calculus, while the last term in curly braces sums to zero as follows.  
Since time in both Cartesian and computational co-ordinates is the same (as was specified in the derivation of the 
metric terms) then $\tau$ = $t$ and thus $\tau_t$ = 1.  Now if the
relation between the Cartesian and computational co-ordinates is held constant in time, then the derivative of the metric 
Jacobian with respect to time must be zero which eliminates the first term in the curly braces (note however, that 
this assumption does not require a grid that is constant in time, only that the relation between the physical and 
computational domains remains fixed).  Substitution of Eqs. \ref{eqn:xit} and \ref{eqn:etat} into the last two terms 
in the curly braces yields,

\begin{displaymath}
	\frac{\partial}{\partial \xi}J^{-1}\Big\{J(-x_t r_\eta + r_t x_\eta)\Big\} 
	+ \frac{\partial}{\partial \eta}J^{-1}\Big\{J(x_t r_\xi - r_t x_\xi)\Big\}
\end{displaymath}

	If we further assume that the derivatives with respect to time are independent of grid location (i.e. changes
in gridding with time are applied uniformly over the entire grid), then $x_t$ and $r_t$
can be pulled outside of the derivatives.  Also, using Clairaut's Theorem which states that

\begin{equation}
	\frac{\partial r^2}{\partial \eta \partial \xi} = \frac{\partial r^2}{\partial \xi \partial \eta}
\label{eqn:clairaut}
\end{equation}

	then the above equation sums to zero.  It should be noted that although restrictions
were placed on the manner in which the grid may change in time, it was not stated that the grid must stay constant in time
(although in most circumstances this will indeed be the case).  Thus Eq. \ref{eqn:Qbig} can be written

\begin{displaymath}
	\frac{1}{J}\frac{\partial Q}{\partial t} = \frac{\partial}{\partial t}(\frac{Q}{J}) 
	+ \frac{\partial}{\partial \xi}(\xi_t \frac{Q}{J}) + \frac{\partial}{\partial \eta}(\eta_t \frac{Q}{J})
\end{displaymath}  

	which when combined with the assertion that the grid spacing is constant in time reduces to 

\begin{equation}
	\frac{1}{J}\frac{\partial Q}{\partial t} = \frac{\partial}{\partial t}(\frac{Q}{J}) 
\label{eqn:Q}
\end{equation}

	It is worth noting here that the above expression in Eq. \ref{eqn:Q} could have been arrived at quite
easily by specifying at the outset a physical domain that remains constant in time, and that time in both the Cartesian
and computational domains is equivalent.

	Considering next the flux vector along the $x$ direction and remembering that one has already divided by the
metric Jacobian, one can write 

\begin{displaymath}
	\frac{1}{J}\frac{\partial E}{\partial x} = \frac{1}{J}\Big[\frac{\partial \xi}{\partial x}\frac{\partial E}{\partial \xi}
	+ \frac{\partial \eta}{\partial x}\frac{\partial E}{\partial \eta}\Big]
\end{displaymath} 

	Adding terms that sum to zero in a fashion similar to that done for the $Q$ vector

\begin{displaymath}
	\frac{1}{J}\frac{\partial E}{\partial x} = \frac{\xi_x}{J}\frac{\partial E}{\partial \xi}
	+\frac{\eta_x}{J}\frac{\partial E}{\partial \eta} + \Big\{E \frac{\partial}{\partial \xi}
	(\frac{\xi_x}{J}) - E\frac{\partial}{\partial \xi}(\frac{\xi_x}{J})\Big\}
	+ \Big\{E\frac{\partial}{\partial \eta}(\frac{\eta_x}{J}) - E\frac{\partial}{\partial \eta}(\frac{\eta_x}{J})\Big\}
\end{displaymath}

	Regrouping terms yields

\begin{displaymath}
	\frac{1}{J}\frac{\partial E}{\partial x} = 
	\Big[\frac{\xi_x}{J}\frac{\partial E}{\partial \xi}
	+ E \frac{\partial}{\partial \xi}(\frac{\xi_x}{J})\Big]
	+ \Big[\frac{\eta_x}{J}\frac{\partial E}{\partial \eta} 
	+ E \frac{\partial}{\partial \eta}(\frac{\eta_x}{J}) \Big]
	-E \Big\{\frac{\partial}{\partial \xi}(\frac{\xi_x}{J})
	+ \frac{\partial}{\partial \eta}(\frac{\eta_x}{J})\Big\}
\end{displaymath}

	The first two groups of terms in square brackets can be combined into a single derivative while the last
term in the curly braces sums to zero by substituting for the definitions of the metric derivatives as follows,

\begin{displaymath}
	-E \Big\{\frac{\partial}{\partial \xi}(\frac{\xi_x}{J})
	+ \frac{\partial}{\partial \eta}(\frac{\eta_x}{J})\Big\} = 
	-E \Big\{\frac{\partial}{\partial \xi}J^{-1}(J r_\eta)
	+ \frac{\partial}{\partial \eta}J^{-1}(J r_\xi)\Big\}
\end{displaymath}

	Again applying Clairaut's Theorem (Eq. \ref{eqn:clairaut}) the above relation reduces to zero.  Therefore, the
derivative of the flux vector $E$ can be expressed as,

\begin{equation}
	\frac{1}{J}\frac{\partial E}{\partial x} = 
	\frac{\partial}{\partial \xi} (\xi_x \frac{E}{J})
	+ \frac{\partial}{\partial \eta} (\eta_x \frac{E}{J}) 
\label{eqn:E}
\end{equation}

	The exact same procedure is followed for the $F$ flux vector resulting in

\begin{equation}
	\frac{1}{J}\frac{\partial F}{\partial r} = 
	\frac{\partial}{\partial \xi} (\xi_r \frac{F}{J})
	+ \frac{\partial}{\partial \eta} (\eta_r \frac{F}{J}) 
\label{eqn:F}
\end{equation}

	Combining the results of Eqs. \ref{eqn:Q}, \ref{eqn:E}, and \ref{eqn:F} into Eq. \ref{eqn:euler} and
realizing that since the axisymmetric source term $S_{be3D}$ is not differentiated it can be left as is yields the
result

\begin{displaymath}
	\frac{\partial}{\partial t}(\frac{Q}{J}) + \frac{\partial}{\partial \xi} \Big[\frac{1}{J}(\xi_x E
	+ \xi_r F)\Big] + \frac{\partial}{\partial \eta}\Big[\frac{1}{J} (\eta_x E 
	+ \eta_r F)\Big] + \frac{1}{J}S_{be3D} = 0
\end{displaymath}

	As a final step, one would like to substitute computational variables for those above, thus following the
conventions established when deriving the metric terms ($x$ = $x_0$, $r$ = $x_1$, $\xi$ = $X_0$, and $\eta$ = $X_1$)
and defining a new variable 

\begin{equation}
	\Omega = J^{-1}
\label{eqn:omega}
\end{equation} 

	one can write

\begin{equation}
	\frac{\partial \bar{Q}}{\partial t} + \frac{\partial \bar{E}}{\partial \xi} 
	+ \frac{\partial \bar{F}}{\partial \eta} + \bar{S}_{be3D} = 0	
\label{eqn:eulerbar}
\end{equation}

	where

\begin{equation}
	\begin{array}{ccc}
		\bar{Q} &=& \Omega Q \\
		\bar{E} &=& \Omega (X_{0,0}E + X_{0,1}F) \\
		\bar{F} &=& \Omega (X_{1,0}E + X_{1,1}F) \\
		\bar{S}_{be3D} &=& \Omega S_{be3D}
	\end{array}
\label{eqn:eulerbars}
\end{equation}

	with $Q$, $E$, $F$, and $S_{be3D}$ defined by Eqs. \ref{eqn:finalQ}, \ref{eqn:finalEF}, and \ref{eqn:finalH} 
respectively.  Extending the same logic to the viscous terms yields the desired form of the governing equations,

\begin{equation}
	\frac{\partial \bar{Q}}{\partial t} + \frac{\partial \bar{E}}{\partial \xi} 
	+ \frac{\partial \bar{F}}{\partial \eta} + \bar{S}_{be3D} -\frac{\partial \bar{E}_v}{\partial \xi} 
	- \frac{\partial \bar{F}_v}{\partial \eta} - \bar{S}_{be3D_v} = 0	
\label{eqn:finalbar}
\end{equation}

	with

\begin{equation}
	\begin{array}{ccc}
		\bar{Q} &=& \Omega Q \\
		\bar{E}_v &=& \Omega (X_{0,0}E_v + X_{0,1}F_v) \\
		\bar{F}_v &=& \Omega (X_{1,0}E_v + X_{1,1}F_v) \\
		\bar{S}_{be3D_v} &=& \Omega S_{be3D_v}
	\end{array}
\label{eqn:viscousbars}
\end{equation}

	However, it must be noted here that the viscous terms themselves contain derivatives and hence
Eqs. \ref{eqn:finalEvFv} and \ref{eqn:finalHv} must be modified to account for the transformation
as well.

%-------------------------------------------------------------------------------------------------------------------
\subsection{Axisymmetric Source Term Transformation}

	Given that without the axisymmetric source terms, the governing equations are simply those for
normal two-dimensional flow, the transformation of the viscous flux vectors will not be demonstrated as these can
be found in other sources.  However of particular interest here are the source terms, which will be transformed
explicitly given that there exist numerous ways in which these terms can be expressed depending on the 
form of the computational governing equations used (i.e. definition of $\Omega$ or  $Q$).

\subsubsection{Inviscid Axisymmetric Source Term}

	Although this term contains no derivatives, it is useful to cast the variables in ones similar in form
to the computational space variables, i.e., similar terms differentiated by subscripts alone.  This would also
apply to the conservative variable vector $Q$ and to the inviscid flux vectors $E$ and $F$ although the process 
will not be shown here.  If we let the physical space velocities $u$ and $v$ be represented by $v_o$ and $v_1$ 
respectively then one can write

\begin{equation}
	\bar{S}_{be3D} = \frac{\Omega}{x_1}\left[ \begin{array}{c}
		\rho_1 v_1 \\
		\vdots \\
		\rho_k v_1 \\
		\rho v_0v_1 \\
		\rho {v_1}^2 \\
		(\rho E + p)v_1 \\
		\rho k v_1 \\
		\rho \omega v_1
		   \end{array}
	    \right]
\label{eqn:finalHbar}
\end{equation}


\subsubsection{Viscous Axisymmetric Source Term}

	Since this term contains derivatives in the Cartesian domain, these quantities must be appropriately
transformed in addition to substituting computational variables for Cartesian ones as was done for the inviscid source
term.  Although the viscous flux vectors require the appropriate transformations as well, given that these vectors
are exactly the same as those for two-dimensional flow, their transformation will not be shown here.  Rewriting the 
viscous source term here for clarity,

\begin{displaymath}
	S_{be3D_v} = \frac{1}{r}\left[ \begin{array}{c}
		\nu_1 \frac{\partial c_1}{\partial r} \\
		\vdots \\
		\nu_k \frac{\partial c_k}{\partial r} \\
		\tau_{rx} - r\frac{2}{3}\frac{\partial}{\partial x}(\mu \frac{v}{r}) \\
		\hat{ \tau}_{rr} -\tau_{\theta \theta} - \frac{2}{3}\mu \frac{v}{r} 
		- r\frac{2}{3}\frac{\partial}{\partial r}(\mu \frac{v}{r}) \\
		\sum_k (\nu_k \frac{\partial c_k}{\partial r}h_k) - q_{c_r} + \tau_{rx}u 
		+ \hat{ \tau}_{rr}v + {\mu_k}^* \frac{\partial k}{\partial r} - \frac{2}{3} \mu \frac{v^2}{r}
			- r\frac{2}{3} \frac{\partial}{\partial x}(\mu \frac{uv}{r})
			- r\frac{2}{3} \frac{\partial}{\partial r}(\mu \frac{v^2}{r}) \\
		{\mu_k}^* \frac{\partial k}{\partial r} + rS_k\\
		{\mu_\omega}^* \frac{\partial \omega}{\partial r} + rS_\omega
		   \end{array}
	    \right]
\end{displaymath}

	Remembering that $r$ = $x_1$ and that $\bar{S}_{be3D_v}$ = $\Omega S_{be3D_v}$ one can immediately
rewrite the above as

\begin{equation}
	\bar{S}_{be3D_v} = \frac{\Omega}{x_1}\left[ \begin{array}{c}
		\nu_1 \frac{\partial c_1}{\partial r} \\
		\vdots \\
		\nu_k \frac{\partial c_k}{\partial r} \\
		\tau_{rx} - x_1\frac{2}{3}\frac{\partial}{\partial x}(\mu \frac{v}{r}) \\
		\hat{ \tau}_{rr} -\tau_{\theta \theta} - \frac{2}{3}\mu \frac{v}{r} 
		- x_1\frac{2}{3}\frac{\partial}{\partial r}(\mu \frac{v}{r}) \\
		\sum_k (\nu_k \frac{\partial c_k}{\partial r}h_k) - q_{c_r} + \tau_{rx}u 
		+ \hat{ \tau}_{rr}v + {\mu_k}^* \frac{\partial k}{\partial r} - \frac{2}{3} \mu \frac{v^2}{r}
			- x_1\frac{2}{3} \frac{\partial}{\partial x}(\mu \frac{uv}{r})
			- x_1\frac{2}{3} \frac{\partial}{\partial r}(\mu \frac{v^2}{r}) \\
		{\mu_k}^* \frac{\partial k}{\partial r} + x_1S_k\\
		{\mu_\omega}^* \frac{\partial \omega}{\partial r} + x_1S_\omega
		   \end{array}
	    \right]
\label{eqn:beginHvbar}
\end{equation}

%---------------------------------------------------------------------------------------------------------------
\subsubsection{Species Continuity Source Term}

	For the remaining transformations, each equation will be considered separately.  Starting with the
species conservation source terms, and letting the number of species vary between 0 and $n_s -1 = \bar{n}_s$
one can write for the first species

\begin{displaymath}
	\nu_0 \frac{\partial}{\partial r}(c_0) = \nu_0 \Big\{\xi_r \frac{\partial c_0}{\partial \xi}
	+ \eta_r \frac{\partial c_0}{\partial \eta}\Big\} = \nu_0 \Big\{X_{0,1} \frac{\partial c_0}{\partial X_0}
	+ X_{1,1} \frac{\partial c_0}{\partial X_1} \Big\}
\end{displaymath}

	Thus for a general species \emph{i} one can write

\begin{equation}
	\nu_i \frac{\partial c_i}{\partial r} = \nu_i \sum_{\theta = 0}^{\bar{n}_d}X_{\theta,1} 
	\frac{\partial c_i}{\partial X_\theta}
\label{eqn:speciescomp}
\end{equation}

	where $\bar{n}_d$ is the number of dimensions minus one, which for axisymmetric flow has a value equal
to 1.  Therefore combining Eq. \ref{eqn:speciescomp} and Eq. \ref{eqn:beginHvbar} yields

\begin{displaymath}
	\bar{S}_{be3D_v} = \frac{\Omega}{x_1}\left[ \begin{array}{c}
		\nu_0 \sum_{\theta = 0}^{\bar{n}_d}X_{\theta,1} 
		\frac{\partial c_0}{\partial X_\theta} \\
		\vdots \\
		\nu_{\bar{n}_s} \sum_{\theta = 0}^{\bar{n}_d}X_{\theta,1} 
		\frac{\partial c_{\bar{n}_s}}{\partial X_\theta} \\
		\vdots
		   \end{array}
	    \right]
\end{displaymath}

%----------------------------------------------------------------------------------------------------------------------
\subsubsection{$x$ Momentum Source Term}

	Considering next the equation of motion along the first physical co-ordinate, by substitution
of Eq. \ref{eqn:taurx} and replacing Cartesian variables one can write

\begin{displaymath}
	\tau_{rx} - x_1\frac{2}{3}\frac{\partial}{\partial x}(\mu \frac{v}{r}) = \mu\Big[\frac{\partial}{\partial x}(v_1) 
	+ \frac{\partial}{\partial r}(v_0)\Big] - x_1\frac{2}{3} \frac{\partial}{\partial x}(\mu \frac{v_1}{x_1})
\end{displaymath}

	Rewriting the Cartesian derivatives using the Calculus as functions of metrics and derivatives along
computational co-ordinate directions yields

\begin{displaymath}
  \begin{array}{c}
	\mu \Big\{\xi_x \frac{\partial v_1}{\partial \xi} + \eta_x \frac{\partial v_1}{\partial \eta} 
	+ \xi_r \frac{\partial v_0}{\partial \xi} + \eta_r \frac{\partial v_0}{\partial \eta}\Big\}
	- x_1 \frac{2}{3} \Big\{\xi_x \frac{\partial}{\partial \xi}(\mu \frac{v_1}{x_1}) 
	+ \eta_x \frac{\partial}{\partial \eta}(\mu \frac{v_1}{x_1}) \Big\}  
	\\
	\mu \Big\{X_{0,0} \frac{\partial v_1}{\partial X_0} + X_{1,0} \frac{\partial v_1}{\partial X_1} 
	+ X_{0,1} \frac{\partial v_0}{\partial X_0} + X_{1,1} \frac{\partial v_0}{\partial X_1}\Big\}
	- x_1 \frac{2}{3} \Big\{X_{0,0} \frac{\partial}{\partial X_0}(\mu \frac{v_1}{x_1}) 
	+ X_{1,0} \frac{\partial}{\partial X_1}(\mu \frac{v_1}{x_1}) \Big\}  
	\\
	\mu \Big\{ \sum_{\theta = 0}^{\bar{n}_d} X_{\theta,1}\frac{\partial v_0}{\partial X_\theta}
	+ \sum_{\theta = 0}^{\bar{n}_d} X_{\theta,0} \frac{\partial v_1}{\partial X_\theta} \Big\}
	- x_1 \frac{2}{3} \sum_{\theta = 0}^{\bar{n}_d} X_{\theta,0} \frac{\partial}{\partial X_\theta}
	(\mu \frac{v_1}{x_1})
  \end{array}
\end{displaymath}

	which can be simplified to 

\begin{equation}
	\tau_{rx} - x_1\frac{2}{3}\frac{\partial}{\partial x}(\mu \frac{v}{r}) =
	\mu \sum_{\theta = 0}^{\bar{n}_d} X_{\theta,1}\frac{\partial v_0}{\partial X_\theta}
	+ \sum_{\theta = 0}^{\bar{n}_d} X_{\theta,0} \Big\{\mu \frac{\partial v_1}{\partial X_\theta} 
	- x_1 \frac{2}{3} \frac{\partial}{\partial X_\theta}(\mu \frac{v_1}{x_1}) \Big\}
\label{eqn:xmomcomp}
\end{equation}

	which when combined with Eqs. \ref{eqn:speciescomp} and Eq. \ref{eqn:beginHvbar} yields

\begin{displaymath}
	\bar{S}_{be3D_v} = \frac{\Omega}{x_1}\left[ \begin{array}{c}
		\nu_0 \sum_{\theta = 0}^{\bar{n}_d}X_{\theta,1} 
		\frac{\partial c_0}{\partial X_\theta} \\
		\vdots \\
		\nu_{\bar{n}_s} \sum_{\theta = 0}^{\bar{n}_d}X_{\theta,1} 
		\frac{\partial c_{\bar{n}_s}}{\partial X_\theta} \\
		\mu \sum_{\theta = 0}^{\bar{n}_d} X_{\theta,1}\frac{\partial v_0}{\partial X_\theta}
		+ \sum_{\theta = 0}^{\bar{n}_d} X_{\theta,0} \Big\{\mu \frac{\partial v_1}{\partial X_\theta} 
		- x_1 \frac{2}{3} \frac{\partial}{\partial X_\theta}(\mu \frac{v_1}{x_1}) \Big\} \\
		\vdots
		   \end{array}
	    \right]
\end{displaymath}

%--------------------------------------------------------------------------------------------------------------
\subsubsection{$r$ Momentum Source Term}

	Considering the momentum equation source term along the second co-ordinate direction and 
substituting the results of Eqs. \ref{eqn:taurrhat} and \ref{eqn:tauthetatheta} in for
the viscous stress terms yields,

\begin{displaymath}
	\hat{ \tau}_{rr} -\tau_{\theta \theta} - \frac{2}{3}\mu \frac{v}{r} 
	- x_1\frac{2}{3}\frac{\partial}{\partial r}(\mu \frac{v}{r}) =  
	\mu ( - \frac{2}{3} \frac{\partial u}{\partial x} + \frac{4}{3} \frac{\partial v}{\partial r})
	- \mu \Big\{- \frac{2}{3} (\frac{\partial u}{\partial x} + \frac{\partial v}{\partial r})  
	+ \frac{4}{3} \frac{v}{r} \Big\} - \frac{2}{3}\mu \frac{v}{r} 
	- x_1\frac{2}{3}\frac{\partial}{\partial r}(\mu \frac{v}{r})
\end{displaymath}

	Rearranging and simplifying results in

\begin{displaymath}
	\mu \Big\{\frac{4}{3}\frac{\partial v}{\partial r} + \frac{2}{3}\frac{\partial v}{\partial r}
	- \frac{2}{3}\frac{\partial u}{\partial x} + \frac{2}{3}\frac{\partial u}{\partial x}  \Big\}
	-\frac{4}{3}(\mu \frac{v}{r}) - \frac{2}{3}(\mu \frac{v}{r}) - x_1\frac{2}{3}\frac{\partial}{\partial r}
	(\mu \frac{v}{r})
\end{displaymath}

\begin{displaymath}
	2\mu \frac{\partial}{\partial r}(v) - x_1\frac{2}{3}\frac{\partial}{\partial r}(\mu \frac{v}{r})
	-2\mu \frac{v}{r}
\end{displaymath}
	
	Rewriting the derivatives and substituting for the computational variables

\begin{displaymath}
  \begin{array}{c}
	2\mu\Big\{\xi_r \frac{\partial v_1}{\partial \xi} + \eta_r \frac{\partial v_1}{\partial \eta} \Big\}
	-x_1\frac{2}{3}\Big\{\xi_r\frac{\partial}{\partial \xi}(\mu\frac{v_1}{x_1}) + \eta_r\frac{\partial}
	{\partial \eta}(\mu\frac{v_1}{x_1})\Big\} - 2\mu \frac{v_1}{x_1}
	\\
	2\mu\Big\{X_{0,1} \frac{\partial v_1}{\partial X_0} + X_{1,1} \frac{\partial v_1}{\partial X_1} \Big\}
	-x_1\frac{2}{3}\Big\{X_{0,1}\frac{\partial}{\partial X_0}(\mu\frac{v_1}{x_1}) + X_{1,1}\frac{\partial}
	{\partial X_1}(\mu\frac{v_1}{x_1})\Big\} - 2\mu \frac{v_1}{x_1}
	\\
	2\mu\Big\{\sum_{\theta = 0}^{\bar{n}_d}X_{\theta,1}\frac{\partial v_1}{\partial X_\theta} 
	-\frac{v_1}{x_1} \Big\}
	-x_1\frac{2}{3}\sum_{\theta = 0}^{\bar{n}_d}X_{\theta,1}\frac{\partial}{\partial X_\theta}(\mu\frac{v_1}{x_1})
  \end{array}
\end{displaymath}

	and thus one can finally write

\begin{equation}
	\hat{ \tau}_{rr} -\tau_{\theta \theta} - \frac{2}{3}\mu \frac{v}{r} 
	- x_1\frac{2}{3}\frac{\partial}{\partial r}(\mu \frac{v}{r}) = 
	\sum_{\theta = 0}^{\bar{n}_d}X_{\theta,1}\Big\{2\mu\frac{\partial v_1}{\partial X_\theta} 
	-x_1\frac{2}{3}\frac{\partial}{\partial X_\theta}(\mu\frac{v_1}{x_1})\Big\} - 2\mu\frac{v_1}{x_1}
\label{eqn:rmomcomp}
\end{equation}

	which when combined with Eqs. \ref{eqn:speciescomp}, \ref{eqn:xmomcomp}, \ref{eqn:rmomcomp}, 
and \ref{eqn:beginHvbar} yields

\begin{displaymath}
	\bar{S}_{be3D_v} = \frac{\Omega}{x_1}\left[ \begin{array}{c}
		\nu_0 \sum_{\theta = 0}^{\bar{n}_d}X_{\theta,1} 
		\frac{\partial c_0}{\partial X_\theta} \\
		\vdots \\
		\nu_{\bar{n}_s} \sum_{\theta = 0}^{\bar{n}_d}X_{\theta,1} 
		\frac{\partial c_{\bar{n}_s}}{\partial X_\theta} \\
		\mu \sum_{\theta = 0}^{\bar{n}_d} X_{\theta,1}\frac{\partial v_0}{\partial X_\theta}
		+ \sum_{\theta = 0}^{\bar{n}_d} X_{\theta,0} \Big\{\mu \frac{\partial v_1}{\partial X_\theta} 
		- x_1 \frac{2}{3} \frac{\partial}{\partial X_\theta}(\mu \frac{v_1}{x_1}) \Big\} \\
		\sum_{\theta = 0}^{\bar{n}_d}X_{\theta,1}\Big\{2\mu\frac{\partial v_1}{\partial X_\theta} 
		-x_1\frac{2}{3}\frac{\partial}{\partial X_\theta}(\mu\frac{v_1}{x_1})\Big\} - 2\mu\frac{v_1}{x_1} \\
		\vdots
		   \end{array}
	    \right]
\end{displaymath}

%----------------------------------------------------------------------------------------------------------------------
\subsubsection{Energy Equation Source Term}

	The next term to be considered is the energy equation source term, which will be broken down into its
individual components given the number of terms that need to be rewritten.  Starting with the species diffusion
contribution and setting $k$ equal to  0 to $\bar{n}_s$

\begin{displaymath}
	\sum_{k = 0}^{\bar{n}_s}(\nu_k \frac{\partial c_k}{\partial r}h_k) = 
	\sum_{k = 0}^{\bar{n}_s}\nu_k h_k \Big\{\xi_r \frac{\partial c_k}{\partial \xi} +
	\eta_r \frac{\partial c_k}{\partial \eta}\Big\} =
	\sum_{k = 0}^{\bar{n}_s}\nu_k h_k \Big\{X_{0,1}\frac{\partial c_k}{\partial X_0} +
	X_{1,1} \frac{\partial c_k}{\partial X_1}\Big\}
\end{displaymath}

	which can be simplified to 

\begin{equation}
	\sum_{k = 0}^{\bar{n}_s}(\nu_k \frac{\partial c_k}{\partial r}h_k) = 
	\sum_{k = 0}^{\bar{n}_s}\nu_k h_k \Big\{\sum_{\theta = 0}^{\bar{n}_d} X_{\theta,1}
	\frac{\partial c_k}{\partial X_\theta}\Big\}
\label{eqn:energyA}
\end{equation}
	
	Next is the heat transfer term, where we will let $k^*$ represent the turbulent coefficient of
thermal conductivity, so as not to confuse it with $k$, the specific turbulent kinetic energy

\begin{displaymath}
	- q_{c_r} = -(-k^* \frac{\partial T}{\partial r}) = k^* \Big\{\xi_r \frac{\partial T}{\partial \xi}
	+ \eta_r \frac{\partial T}{\partial \eta}\Big\} = k^*\Big\{X_{0,1} \frac{\partial T}{\partial X_0}
	+ X_{1,1} \frac{\partial T}{\partial X_1}\Big\}
\end{displaymath}

	where it is noted that Eq. \ref{eqn:rheatflux} was used to describe the heat flux term.  Simplifying
the above yields,

\begin{equation}
	- q_{c_r} = k^*\sum_{\theta = 0}^{\bar{n}_d} X_{\theta,1} \frac{\partial T}{\partial X_\theta}
\label{eqn:energyB}
\end{equation}
	
	Considering next the first stress term $\tau_{rx}$ and using the results of Eq. \ref{eqn:taurx}

\begin{displaymath}
	\tau_{rx}u = u\Big\{\mu\Big[\frac{\partial}{\partial x}(v) + \frac{\partial}{\partial r}(u)\Big]\Big\} =
	u\mu\Big\{\xi_x \frac{\partial v}{\partial \xi} + \eta_x \frac{\partial v}{\partial \eta} +
	\xi_r \frac{\partial u}{\partial \xi} + \eta_r \frac{\partial u}{\partial \eta} \Big\} 
\end{displaymath}

\begin{displaymath}
	\tau_{rx}u = 
	v_0 \mu \Big\{X_{0,0} \frac{\partial v_1}{\partial X_0} + X_{1,0} \frac{\partial v_1}{\partial X_1} +
	X_{0,1} \frac{\partial v_0}{\partial X_0} + X_{1,1} \frac{\partial v_0}{\partial X_1} \Big\}
\end{displaymath}

	which can be simplified to

\begin{equation}
	\tau_{rx}u = v_0\mu
	\Big\{\sum_{\theta = 0}^{\bar{n}_d} X_{\theta,1}\frac{\partial v_0}{\partial X_\theta} +
	\sum_{\theta = 0}^{\bar{n}_d} X_{\theta,0}\frac{\partial v_1}{\partial X_\theta}\Big\}
\label{eqn:energyC}
\end{equation}

	The second stress term can be written with the help of Eq. \ref{eqn:taurrhat} as
	
\begin{displaymath}
	\hat{ \tau}_{rr}v = v\Big\{\mu\Big[ -\frac{2}{3} \frac{\partial}{\partial x}(u) 
	+ \frac{4}{3} \frac{\partial}{\partial r}(v)\Big]\Big\} = v\mu\Big\{-\frac{2}{3}\Big[
	\xi_x\frac{\partial u}{\partial \xi} + \eta_x\frac{\partial u}{\partial \eta}\Big] + \frac{4}{3}\Big[
	\xi_r\frac{\partial v}{\partial \eta} + \eta_r\frac{\partial v}{\partial \eta}\Big] \Big\}
\end{displaymath}

\begin{displaymath}
	\hat{ \tau}_{rr}v = v_1\mu\Big\{-\frac{2}{3}\Big[
	X_{0,0}\frac{\partial v_0}{\partial X_0} + X_{1,0}\frac{\partial v_0}{\partial X_1}\Big] + \frac{4}{3}\Big[
	X_{0,1}\frac{\partial v_1}{\partial X_0} + X_{1,1}\frac{\partial v_1}{\partial X_1}\Big] \Big\} 
\end{displaymath}

	which can be simplified to

\begin{equation}
	\hat{ \tau}_{rr}v = v_1\mu\Big\{-\frac{2}{3}\sum_{\theta = 0}^{\bar{n}_d}X_{\theta,0}
	\frac{\partial v_0}{\partial X_\theta} + \frac{4}{3}\sum_{\theta = 0}^{\bar{n}_d}X_{\theta,1}
	\frac{\partial v_1}{\partial X_\theta} \Big\}	
\label{eqn:energyD}
\end{equation}

	The turbulent energy diffusion term is similar to the heat transfer term

\begin{displaymath}
	{\mu_k}^* \frac{\partial k}{\partial r} = {\mu_k}^* \Big\{\xi_r\frac{\partial k}{\partial \xi} 
	+ \eta_r\frac{\partial k}{\partial \eta}\Big\} = {\mu_k}^* \Big\{X_{0,1}\frac{\partial k}{\partial X_0}
	+ X_{1,1}\frac{\partial k}{\partial X_1}\Big\}
\end{displaymath}	

	which is simplified to

\begin{equation}
	{\mu_k}^* \frac{\partial k}{\partial r} = {\mu_k}^* \Big\{\sum_{\theta = 0}^{\bar{n}_d}X_{\theta,1}
	\frac{\partial k}{\partial X_\theta}\Big\}
\label{eqn:energyE}
\end{equation}

	The next term has no derivative terms and hence requires only direct substitution for the computational 
variables

\begin{equation}
	- \frac{2}{3}\mu\frac{v^2}{r} = - \frac{2}{3}\mu\frac{{v_1}^2}{x_1}
\label{eqn:energyF}
\end{equation}

	The last two terms are treated as follows

\begin{displaymath}
	- x_1\frac{2}{3}\frac{\partial}{\partial x}(\mu \frac{uv}{r}) = -x_1\frac{2}{3}\Big\{\xi_x
	\frac{\partial}{\partial \xi}(\mu\frac{v_0 v_1}{x_1}) + \eta_x\frac{\partial}{\partial \eta}
	(\mu\frac{v_0 v_1}{x_1})\Big\} = -x_1\frac{2}{3}\Big\{X_{0,0}
	\frac{\partial}{\partial X_0}(\mu\frac{v_0 v_1}{x_1}) + X_{1,0}\frac{\partial}{\partial X_1}
	(\mu\frac{v_0 v_1}{x_1})\Big\}	
\end{displaymath}

\begin{displaymath}
	- x_1\frac{2}{3}\frac{\partial}{\partial r}(\mu \frac{v^2}{r}) = -x_1\frac{2}{3}\Big\{\xi_r
	\frac{\partial}{\partial \xi}(\mu\frac{{v_1}^2}{x_1}) + \eta_r\frac{\partial}{\partial \eta}(\mu\frac{{v_1}^2}{x_1})\Big\}
	= -x_1\frac{2}{3}\Big\{X_{0,1}
	\frac{\partial}{\partial X_0}(\mu\frac{{v_1}^2}{x_1}) + X_{1,1}\frac{\partial}{\partial X_1}(\mu\frac{{v_1}^2}{x_1})\Big\}
\end{displaymath}

	which can combined and rewritten as

\begin{equation}
	- x_1\frac{2}{3}\frac{\partial}{\partial x}(\mu \frac{uv}{r}) - x_1\frac{2}{3}\frac{\partial}{\partial r}
	(\mu \frac{v^2}{r}) = -x_1\frac{2}{3}\Big\{\sum_{\theta = 0}^{\bar{n}_d} \sum_{\vartheta = 0}^{\bar{n}_d}
	X_{\theta,\vartheta}\frac{\partial}{\partial X_\theta}(\mu\frac{v_\vartheta v_1}{x_1}) \Big\}
\label{eqn:energyG}
\end{equation}

	At this point all the individual terms have been rewritten using the appropriate metric terms and hence the
axisymmetric energy equation source term can be expressed as a combination of Eqs. \ref{eqn:energyA} - \ref{eqn:energyG}

\begin{equation}
  \begin{array}{c}
	\sum_{k = 0}^{\bar{n}_s}\nu_k h_k \Big\{\sum_{\theta = 0}^{\bar{n}_d} X_{\theta,1}
	\frac{\partial c_k}{\partial X_\theta}\Big\} + k^*\sum_{\theta = 0}^{\bar{n}_d} X_{\theta,1} 
	\frac{\partial T}{\partial X_\theta} + v_0\mu\Big\{\sum_{\theta = 0}^{\bar{n}_d} X_{\theta,1}
	\frac{\partial v_0}{\partial X_\theta} + \sum_{\theta = 0}^{\bar{n}_d} X_{\theta,0}
	\frac{\partial v_1}{\partial X_\theta}\Big\} \\
	+ v_1\mu\Big\{-\frac{2}{3}\sum_{\theta = 0}^{\bar{n}_d}X_{\theta,0}
	\frac{\partial v_0}{\partial X_\theta} + \frac{4}{3}\sum_{\theta = 0}^{\bar{n}_d}X_{\theta,1}
	\frac{\partial v_1}{\partial X_\theta} \Big\} 
	+ {\mu_k}^* \Big\{\sum_{\theta = 0}^{\bar{n}_d}X_{\theta,1}
	\frac{\partial k}{\partial X_\theta}\Big\} \\
	- \frac{2}{3}\mu\frac{{v_1}^2}{x_1}
	 -x_1\frac{2}{3}\Big\{\sum_{\theta = 0}^{\bar{n}_d} \sum_{\vartheta = 0}^{\bar{n}_d}
	X_{\theta,\vartheta}\frac{\partial}{\partial X_\theta}(\mu\frac{v_\vartheta v_1}{x_1}) \Big\}
  \end{array}
\label{eqn:energycomp}
\end{equation}

	which when combined with Eqs. \ref{eqn:speciescomp}, \ref{eqn:xmomcomp}, \ref{eqn:rmomcomp}, 
and \ref{eqn:beginHvbar} yields 

\begin{displaymath}
	\bar{S}_{be3D_v} = \frac{\Omega}{x_1}\left[ \begin{array}{c}
		\nu_0 \sum_{\theta = 0}^{\bar{n}_d}X_{\theta,1} 
		\frac{\partial c_0}{\partial X_\theta} \\
		\vdots \\
		\nu_{\bar{n}_s} \sum_{\theta = 0}^{\bar{n}_d}X_{\theta,1} 
		\frac{\partial c_{\bar{n}_s}}{\partial X_\theta} 
		\\
		\mu \sum_{\theta = 0}^{\bar{n}_d} X_{\theta,1}\frac{\partial v_0}{\partial X_\theta}
		+ \sum_{\theta = 0}^{\bar{n}_d} X_{\theta,0} \Big\{\mu \frac{\partial v_1}{\partial X_\theta} 
		- x_1 \frac{2}{3} \frac{\partial}{\partial X_\theta}(\mu \frac{v_1}{x_1}) \Big\} 
		\\
		\sum_{\theta = 0}^{\bar{n}_d}X_{\theta,1}\Big\{2\mu\frac{\partial v_1}{\partial X_\theta} 
		-x_1\frac{2}{3}\frac{\partial}{\partial X_\theta}(\mu\frac{v_1}{x_1})\Big\} - 2\mu\frac{v_1}{x_1} 
		\\
	\left(\begin{array}{c}
		\sum_{\theta = 0}^{\bar{n}_d}X_{\theta,0} \Big\{v_0\mu\frac{\partial v_1}{\partial X_\theta} -\frac{2}{3}
		v_1\mu\frac{\partial v_0}{\partial X_\theta} \Big\} + \sum_{\theta = 0}^{\bar{n}_d}X_{\theta,1}\Big\{
		\sum_{k = 0}^{\bar{n}_s}\nu_k h_k \frac{\partial c_k}{\partial X_\theta} + 
		k^*\frac{\partial T}{\partial X_\theta} + {\mu_k}^*\frac{\partial k}{\partial X_\theta} \\ 
		+ v_0 \mu \frac{\partial v_0}{\partial X_\theta} + \frac{4}{3}v_1 \mu \frac{\partial v_1}{\partial X_\theta}
		\Big\} -\frac{2}{3}\Big\{\mu\frac{{v_1}^2}{x_1} + x_1\sum_{\theta = 0}^{\bar{n}_d}\sum_{\vartheta = 0}^{\bar{n}_d}
		X_{\theta,\vartheta}(\mu\frac{v_\vartheta v_1}{x_1})\Big\}
	\end{array}\right)
		\\	
		\vdots
		   \end{array}
	    \right]
\end{displaymath}

%--------------------------------------------------------------------------------------------------------------------------
\subsubsection{Turbulent Source Terms}

	Examining the remaining source terms for the turbulence quantities one can write

\begin{displaymath}
	{\mu_k}^* \frac{\partial k}{\partial r} + x_1S_k = {\mu_k}^* \Big\{\xi_r \frac{\partial k}{\partial \xi}
	+\eta_r \frac{\partial k}{\partial \eta}\Big\} + x_1S_k = {\mu_k}^* \Big\{X_{0,1} \frac{\partial k}{\partial X_0}
	+ X_{1,1} \frac{\partial k}{\partial X_1}\Big\} + x_1S_k
\end{displaymath}	

\begin{displaymath}
	{\mu_\omega}^* \frac{\partial \omega}{\partial r} + x_1S_\omega = {\mu_\omega}^* 
	\Big\{\xi_r \frac{\partial \omega}{\partial \xi}
	+\eta_r \frac{\partial \omega}{\partial \eta}\Big\} + x_1S_\omega = {\mu_\omega}^* 
	\Big\{X_{0,1} \frac{\partial \omega}{\partial X_0}
	+ X_{1,1} \frac{\partial \omega}{\partial X_1}\Big\} + x_1S_\omega
\end{displaymath}	

	both of which can be simplified similarly to

\begin{equation}
	{\mu_k}^*\sum_{\theta = 0}^{\bar{n}_d}X_{\theta,1}\frac{\partial k}{\partial X_\theta} + x_1S_k
\label{eqn:kcomp}
\end{equation}

\begin{equation}
	{\mu_\omega}^*\sum_{\theta = 0}^{\bar{n}_d}X_{\theta,1}\frac{\partial \omega}{\partial X_\theta} + x_1S_\omega
\label{eqn:omegacomp}
\end{equation}

	As a last step, since both $S_k$ and $S_\omega$ contain the variable $P_k$ which itself contains an axisymmetric
term (Eq. \ref{eqn:pkbe3d}), this term must be transformed as well.  Starting from the definition of $P_{k_be3D}$ (where
the $\tilde v$ has been removed for simplicity but is still understood),

\begin{displaymath}
	P_{k_{be3D}} = -\mu_T\frac{2}{3}\frac{v}{r}(\frac{\partial u}{\partial x} + \frac{\partial v}{\partial r})
\end{displaymath}

	Making the substitution for $u$ and $v$ into the computational space variables $v_0$ and $v_1$ respectively while at the 
same time replacing the physical space variables in a similar fashion yields,

\begin{displaymath}
	P_{k_{be3D}} = -\mu_T\frac{2}{3}\frac{v_1}{x_1}(\frac{\partial v_0}{\partial x_0} + \frac{\partial v_1}{\partial x_1})
\end{displaymath}

	Using the Calculus to rewrite the Cartesian derivatives in terms of metrics and derivatives along the computational
co-ordinates,

\begin{displaymath}
	P_{k_{be3D}} = -\mu_T\frac{2}{3}\frac{v_1}{x_1}\Big\{\xi_x \frac{\partial v_o}{\partial \xi} +
	\eta_x \frac{\partial v_o}{\partial \eta} + \xi_r \frac{\partial v_1}{\partial \xi} + \eta_r
	\frac{\partial v_1}{\partial \eta}\Big\}
\end{displaymath}

\begin{displaymath}
	= -\mu_T\frac{2}{3}\frac{v_1}{x_1}\Big\{X_{0,0} \frac{\partial v_o}{\partial X_O} +
	X_{1,0} \frac{\partial v_o}{\partial X_1} + X_{0,1} \frac{\partial v_1}{\partial X_0} + X_{1,1}
	\frac{\partial v_1}{\partial X_1}\Big\}
\end{displaymath}

	which can be simplified in summation form to,

\begin{equation}
	P_{k_{be3D}}= -\mu_T\frac{2}{3}\frac{v_1}{x_1}\sum_{\theta=0}^{\theta=1}\sum_{\vartheta=0}^{\vartheta=1}
	X_{\theta,\vartheta}\frac{\partial v_{\vartheta}}{\partial X_{\theta}}
\label{eqn:pkbe3dtrans}
\end{equation}

%----------------------------------------------------------------------------------------------------------------------
\subsubsection{Summary of Axisymmetric Source Terms}

	Therefore the final form of the axisymmetric, multi-species, viscous, turbulent source term can be written as
the combination of Eqs. \ref{eqn:speciescomp}, \ref{eqn:xmomcomp}, \ref{eqn:rmomcomp}, \ref{eqn:kcomp}, and 
\ref{eqn:omegacomp} 
 
\begin{equation}
	\bar{S}_{be3D_v} = \frac{\Omega}{x_1}\left[ \begin{array}{c}
		\nu_0 \sum_{\theta = 0}^{\bar{n}_d}X_{\theta,1} 
		\frac{\partial c_0}{\partial X_\theta} \\
		\vdots \\
		\nu_{\bar{n}_s} \sum_{\theta = 0}^{\bar{n}_d}X_{\theta,1} 
		\frac{\partial c_{\bar{n}_s}}{\partial X_\theta} 
		\\
		\mu \sum_{\theta = 0}^{\bar{n}_d} X_{\theta,1}\frac{\partial v_0}{\partial X_\theta}
		+ \sum_{\theta = 0}^{\bar{n}_d} X_{\theta,0} \Big\{\mu \frac{\partial v_1}{\partial X_\theta} 
		- x_1 \frac{2}{3} \frac{\partial}{\partial X_\theta}(\mu \frac{v_1}{x_1}) \Big\} 
		\\
		\sum_{\theta = 0}^{\bar{n}_d}X_{\theta,1}\Big\{2\mu\frac{\partial v_1}{\partial X_\theta} 
		-x_1\frac{2}{3}\frac{\partial}{\partial X_\theta}(\mu\frac{v_1}{x_1})\Big\} - 2\mu\frac{v_1}{x_1} 
		\\
	\left(\begin{array}{c}
		\sum_{\theta = 0}^{\bar{n}_d}X_{\theta,0} \Big\{v_0\mu\frac{\partial v_1}{\partial X_\theta} -\frac{2}{3}
		v_1\mu\frac{\partial v_0}{\partial X_\theta} \Big\} + \sum_{\theta = 0}^{\bar{n}_d}X_{\theta,1}\Big\{
		\sum_{k = 0}^{\bar{n}_s}\nu_k h_k \frac{\partial c_k}{\partial X_\theta} + 
		k^*\frac{\partial T}{\partial X_\theta} + {\mu_k}^*\frac{\partial k}{\partial X_\theta} \\ 
		+ v_0 \mu \frac{\partial v_0}{\partial X_\theta} + \frac{4}{3}v_1 \mu \frac{\partial v_1}{\partial X_\theta}
		\Big\} -\frac{2}{3}\Big\{\mu\frac{{v_1}^2}{x_1} + x_1\sum_{\theta = 0}^{\bar{n}_d}\sum_{\vartheta = 0}^{\bar{n}_d}
		X_{\theta,\vartheta}(\mu\frac{v_\vartheta v_1}{x_1})\Big\}
	\end{array}\right)
		\\	
		{\mu_k}^*\sum_{\theta = 0}^{\bar{n}_d}X_{\theta,1}\frac{\partial k}{\partial X_\theta} + x_1S_k 
		\\
		{\mu_\omega}^*\sum_{\theta = 0}^{\bar{n}_d}X_{\theta,1}\frac{\partial \omega}{\partial X_\theta} + x_1S_\omega
		   \end{array}
	    \right]
\label{finalHvbar}
\end{equation} 

	while the inviscid source term is defined by Eq. \ref{eqn:finalHbar}

\begin{displaymath}
	\bar{S}_{be3D} = \frac{\Omega}{x_1}\left[ \begin{array}{c}
		\rho_1 v_1 \\
		\vdots \\
		\rho_k v_1 \\
		\rho v_0v_1 \\
		\rho {v_1}^2 \\
		(\rho E + p)v_1 \\
		\rho k v_1 \\
		\rho \omega v_1
		   \end{array}
	    \right]
\end{displaymath}