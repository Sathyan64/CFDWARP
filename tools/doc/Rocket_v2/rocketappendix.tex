\section{Two Dimensional Method of Characteristics (Irrotational Flow)}

\subsection{Derivation of Characteristic and Compatibility Equations}

	The gas dynamic equation (Eq. \ref{eqn:gasdynq}), stated again here for convenience,
can be rewritten as follows.  

\begin{displaymath}
	\vec{q}\cdot\nabla(\frac{q^2}{2}) - a^2(\nabla \cdot \vec{q}) = 0 	
\end{displaymath}

	Taking $\vec{q} = u + v$ where $u$ and $v$ are the velocities parallel and normal to the axisymmetric 
axis respectively, then the dot product $\nabla \cdot \vec{q}$ can be expressed in cylindrical co-ordinates as,

\begin{equation}
	\nabla\cdot\vec{q} = \frac{\partial u}{\partial x} + \frac{\partial v}{\partial r} 
	+ \frac{v}{r}
\label{eqn:dotprod}
\end{equation}

	Substituting this into the gas dynamic equation yields,

\begin{displaymath}
	\vec{q}\cdot\nabla(\frac{q^2}{2}) - a^2\Big\{\frac{\partial u}{\partial x} + \frac{\partial v}
	{\partial r} + \frac{v}{r}\Big\} = 0 	
\end{displaymath}

	Expanding out the first term one can write,

\begin{displaymath}
	(u+v)\cdot\Big\{\frac{1}{2}(\frac{\partial q^2}{\partial x} + \frac{\partial q^2}{\partial r})\Big\}
	- a^2\Big\{\frac{\partial u}{\partial x} + \frac{\partial v}
	{\partial r} + \frac{v}{r}\Big\}=\frac{1}{2}(u\frac{\partial q^2}{\partial x} + v\frac{\partial q^2}
	{\partial r})- a^2\Big\{\frac{\partial u}{\partial x} + \frac{\partial v}
	{\partial r} + \frac{v}{r}\Big\} = 0 	
\end{displaymath}

	Given the definition of $q$ one can write for the derivative of its square,

\begin{displaymath}
	\frac{\partial q^2}{\partial x} = \frac{\partial}{\partial x}(u^2 + v^2) = 
	2(u\frac{\partial u}{\partial x} + v\frac{\partial v}{\partial x})
\end{displaymath}

	and

\begin{displaymath}
	\frac{\partial q^2}{\partial r} = \frac{\partial}{\partial r}(u^2 + v^2) = 
	2(u\frac{\partial u}{\partial r} + v\frac{\partial v}{\partial r})
\end{displaymath}

	which when substituted into the gas dynamic equation yields,

\begin{displaymath}
	u(u\frac{\partial u}{\partial x} + v\frac{\partial v}{\partial x}) + 
	v(u\frac{\partial u}{\partial r} + v\frac{\partial v}{\partial r}) 
	- a^2(\frac{\partial u}{\partial x} + \frac{\partial v}
	{\partial r} + \frac{v}{r}) = 0 	
\end{displaymath}

	If the flow is assumed irrotational then one also write the relation

\begin{equation}
	\frac{du}{dr} - \frac{dv}{dx} = 0
\label{eqn:irrotational}
\end{equation}

	which allows the above to be simplified to,

\begin{equation}
	(u^2-a^2)\frac{du}{dx} + (v^2-a^2)\frac{dv}{dr} + 2uv\frac{du}{dr} 
	- \frac{a^2v}{r} = 0
\label{eqn:gasdynaxi}
\end{equation}

	Equations \ref{eqn:irrotational} and \ref{eqn:gasdynaxi} form the basis of the two dimensional axisymmetric
method of characteristics,
as from these one can develop both the compatibility and characteristic equations.  Starting
by multiplying Eq. \ref{eqn:gasdynaxi} by $\sigma_1$ and Eq. \ref{eqn:irrotational} by
$\sigma_2$ (where $\sigma_1$ and $\sigma_2$ are arbitrary functions to be determined later)
and summing the resulting equations to zero yields,

\begin{displaymath}	
	\sigma_1(u^2-a^2)\frac{du}{dx} + \sigma_1(v^2-a^2)\frac{dv}{dr} + \sigma_1 2uv\frac{du}{dr} 
	- \sigma_1\frac{a^2v}{r} + \sigma_2\frac{du}{dr} - \sigma_2\frac{dv}{dx} = 0
\end{displaymath}

	which after grouping the derivatives of a single velocity together and rearranging yields,

\begin{equation}
	\sigma_1(u^2-a^2)\Big\{\frac{\partial u}{\partial x} + \frac{\sigma_1 2uv + \sigma_2}
	{\sigma_1(u^2-a^2)}\frac{\partial u}{\partial r}\Big\} - \sigma_2\Big\{\frac{\partial v}
	{\partial x} + \frac{\sigma_1(v^2-a^2)}{-\sigma_2}\frac{\partial v}{\partial r}\Big\}
	- \sigma_1\frac{a^2v}{r} = 0
\label{eqn:big}
\end{equation}

	Since one of the properties enforced on the dependent variables ($u(x,y)$ and $v(x,y)$)  
is that they be continuous functions across the characteristic curves one can write,

\begin{displaymath}
	\frac{du}{dx} = \frac{\partial u}{\partial x} + \lambda\frac{\partial u}{\partial r}
	\hspace{2cm}\textrm{and}\hspace{2cm} \frac{dv}{dx} = \frac{\partial v}{\partial x} + 
	\lambda\frac{\partial v}{\partial r}
\end{displaymath}

	where $\lambda = \frac{dy}{dx}$ and is hence the slope of the characteristic in the
physical plane along which the total derivatives can be defined as above.  Therefore the 
characteristic curves for Eq. \ref{eqn:big} must have slopes of,

\begin{equation}
	\lambda = \frac{\sigma_1 2uv + \sigma_2}{\sigma_1(u^2-a^2)} \hspace{2cm}\textrm{and}\hspace{2cm}
	\lambda = \frac{\sigma_1(v^2-a^2)}{-\sigma_2}
\label{eqn:lambdasig}
\end{equation}

	which then allows Eq. \ref{eqn:big} to be written as,

\begin{equation}
	\sigma_1(u^2-a^2)du - \sigma_2 dv - \sigma_1\frac{a^2v}{r}dx = 0
\label{eqn:small}
\end{equation}

	Equation \ref{eqn:small} is the compatibility equation for axisymmetric irrotational flow,
as defined by Eqs. \ref{eqn:irrotational} and \ref{eqn:gasdynaxi}.  However, for this equation to 
be of use, the arbitrary functions must be phrased in terms of known quantities.  Therefore,
rewriting Eqs. \ref{eqn:lambdasig} in matrix form gives,

\begin{displaymath}
	\left[\begin{array}{cc}
	(u^2-a^2)\lambda - 2uv & -1 \\ (v^2-a^2) & \lambda
	\end{array}\right] 
	\left[\begin{array}{c}
	\sigma_1 \\ \sigma_2
	\end{array}\right] = 0 
\end{displaymath}

	Taking the determinant of the co-efficient matrix and setting this equal to zero yields,

\begin{displaymath}
	(u^2-a^2)\lambda^2 -2uv\lambda + (v^2-a^2) = 0 
\end{displaymath}

	which can be solved using the quadratic formula to give the result,

\begin{equation}
	\lambda_\pm = \frac{uv \pm a^2\sqrt{M^2-1}}{u^2-a^2}
\label{eqn:lambdamach}
\end{equation}

	where the $+$ sign refers to the positive square root and the $-$ refers to the negative.
Since $\lambda$ is the slope of the characteristic in the physical plane, this means that for flows
where $M > 1$ there exist two distinct characteristic curves.  Equation \ref{eqn:lambdamach} can
be rephrased by noting that,

\begin{displaymath}
	\theta  = \tan^{-1}(\frac{v}{u}) \hspace{1.5cm} \textrm{,} \hspace{1.5cm} 
	u = q\cos\theta \hspace{1.5cm} \textrm{,} \hspace{1.5cm} v = q\sin\theta 
\end{displaymath}

	while from the Mach triangle one can write (where $\alpha$ is the Mach angle),

\begin{displaymath}
	\alpha = \sin^{-1}(\frac{1}{M}) \hspace{1.5cm} \textrm{,} \hspace{1.5cm}
	\tan\alpha = \frac{1}{\sqrt{M^2-1}}
\end{displaymath}

	Substituting these relations into Eq. \ref{eqn:lambdamach} yields,

\begin{displaymath}
	\lambda_\pm = \frac{(q\cos\theta) (q\sin\theta) \pm a^2(\frac{1}{\tan\alpha})}
	{(q\cos\theta)^2-a^2}
\end{displaymath}	

	while dividing through by $q^2$ and noting that $\frac{a^2}{q^2} = M^{-2}$ yields,

\begin{displaymath}
	\lambda_\pm = \frac{\cos\theta \sin\theta \pm \frac{1}{\tan\alpha}M^{-2}}
	{\cos^2\theta-M^{-2}}
\end{displaymath}	

	Using the definition of the Mach angle allows the above to be simplified to,

\begin{displaymath}
	\lambda_\pm = \frac{\cos\theta \sin\theta \pm \cos\alpha \sin\alpha}
	{\cos^2\theta-\sin^2\alpha}
\end{displaymath}

	The above can be further simplified by making use of trigonometric identities
allowing an alternate form for $\lambda$ to be expressed as,

\begin{equation}
	\fbox{$
	\lambda_\pm = (\frac{dy}{dx})_\pm = \tan(\theta \pm \alpha)
	$}
\label{eqn:lambda}
\end{equation}

	Equation \ref{eqn:lambda} is the characteristic equation for axisymmetric irrotational
flow.  Having determined the characteristic equation one can return to the compatibility equation
to eliminate the arbitrary functions $\sigma_1$ and $\sigma_2$ as follows.  From Eqs. 
\ref{eqn:lambdasig} one can express $\sigma_2$ as,

\begin{displaymath}
	\sigma_1\Big\{(u^2-a^2)\lambda -2uv\Big\} = \sigma_2 = -\frac{\sigma_1(v^2-a^2)}{\lambda}
\end{displaymath}
 
	Although not immediately apparent, these two equations are not independent and thus either
one can be substituted back into Eq. \ref{eqn:small} to eliminate both arbitrary functions,

\begin{displaymath}
	\sigma_1(u^2-a^2)du - \Big\{\sigma_1\Big[(u^2-a^2)\lambda -2uv\Big]\Big\}dv 
	- \sigma_1\frac{a^2v}{r}dx = 0
\end{displaymath}

	which after dividing through by $\sigma_1$ yields,

\begin{equation}
	\fbox{$
	(u^2-a^2)du_{\pm} - \Big[(u^2-a^2)\lambda_{\pm} -2uv\Big]dv_{\pm} - \frac{a^2v}{r}dx_{\pm} = 0
	$}
\label{eqn:compatibility}
\end{equation}

	where if the positive value of $\lambda$ is used, then Eq. \ref{eqn:compatibility}
applies along the $C_+$ characteristic and hence all values $du$, $dv$, and $dx$ apply to
this curve.  It is noted that if the other definition of $\sigma_2$ is substituted into
Eq. \ref{eqn:small} to eliminate the arbitrary values, then a slightly different form of the
compatibility equation is obtained.  However, this merely produces an alternate form of the
same equation, not an independent relation that is also applicable along the characteristic.

%--------------------------------------------------------------------------------------------
\subsection{Numerical Implementation of the Method of Characteristics}

	Having derived the characteristic (Eq. \ref{eqn:lambda}) and compatibility (Eq. 
\ref{eqn:compatibility}) equations, one can now use these in a numerical integration 
scheme to completely determine a supersonic flow field.  If one defines the following 
parameters,

\begin{equation}
	Q = u^2 - a^2 \hspace {1.5cm} R = 2uv - Q\lambda \hspace{1.5cm} S = \frac{a^2v}{y}
\label{eqn:variables}
\end{equation}

	then the characteristic and compatibility equations can be written in finite difference
form as,

\begin{equation}
	\Delta y_{\pm} = \lambda_{\pm} \Delta x_{\pm}
\label{eqn:charfinite}
\end{equation}

\begin{equation}
	Q_{\pm}\Delta u_{\pm} + R_{\pm}\Delta v_{\pm} - S_{\pm}\Delta x_{\pm} = 0
\label{eqn:compfinite}
\end{equation}

	If one considers a line along which the flow field in completely known, then from 
two points (points 1 and 2) along this line, two opposing characteristic curves will intersect 
each other at some unknown point 4.  Using these three points, one can use Eq. \ref{eqn:charfinite}
to determine the intersection point (4) as follows,

\begin{displaymath}
	y_4 = y_1 + \lambda_-(x_4 - x_1)
\end{displaymath}

	while from the other characteristic,

\begin{equation}
	\fbox{$
	y_4 = y_2 + \lambda_+(x_4 - x_2) 
	$}
\label{eqn:y4}
\end{equation}

	thus yielding for $x_4$,

\begin{equation}
	\fbox{$
	x_4 = \frac{y_1 - y_2 - \lambda_-x_1 + \lambda_+x_2}{\lambda_+ - \lambda_-}
	$}
\label{eqn:x4}
\end{equation}

	Once $x_4$ is found, expression in Eq. \ref{eqn:y4} can be used to completely
determine the physical location of the intersection point of the two opposing characteristic
curves emanating from points 1 and 2.  To determine the flow properties at this point one
can make use of Eq. \ref{eqn:compfinite} by calculating along each characteristic,

\begin{equation}
	\begin{array}{c}
	T_{-} = S_{-}(x_4 - x_1) + Q_{-}u_1 + R_{-}v_1 \\ \vspace{0.5cm}
	T_{+} = S_{+}(x_4 - x_2) + Q_{+}u_2 + R_{+}v_2
	\end{array}
\label{eqn:tpm}
\end{equation}

	which then from Eq. \ref{eqn:compfinite} along the $C_-$ characteristic,

\begin{displaymath}
	Q_-(u_4 - u_1) + R_-(v_4 - v_1) -S_-(x_4 - x_1) = 0
\end{displaymath}

	while rearranging and substituting Eq. \ref{eqn:tpm} yields,

\begin{displaymath}
	Q_-u_4 + R_-v_4 = S_-(x_4 - x_1) + Q_-u_1 + R_-v_1 = T_-
\end{displaymath}

	therefore,

\begin{equation}
	\fbox{$
	v_4 = \frac{T_- - Q_-u_4}{R_-}
	$}
\label{eqn:v4}
\end{equation}

	Along the $C_+$ characteristic one can write,

\begin{displaymath}
	Q_+u_4 + R_+v_4 = S_+(x_4 - x_2) + Q_+u_2 + R_+v_2 = T_+ 
\end{displaymath}

	which when Eq. \ref{eqn:v4} is used to replace $v_4$ yields a solution for $u_4$,

\begin{displaymath}
	Q_+u_4 + R_+\Big\{\frac{T_- - Q_-u_4}{R_-}\Big\} = T_+ 
\end{displaymath}

\begin{displaymath}
	R_+T_- - R_+Q_-u_4 = R_-T_+ - R_-Q_+u_4 
\end{displaymath}

\begin{equation}
	\fbox{$
	u_4 = \frac{R_+T_- - R_-T_+}{R_+Q_- -  R_-Q_+}
	$}
\label{eqn:u4}
\end{equation}

	with the velocity components found from Eqs. \ref{eqn:v4} and \ref{eqn:u4} at
the intersection point 4, one has now determined the flow conditions at a
point further downstream (or upstream, depending on the direction chosen) of the original
line at which the flow conditions are known (noting that each subsequent intersection point
must still lie within a supersonic flow field).  

	The accuracy of the characteristic mesh being 
built can be improved by employing a predictor-corrector scheme to the solution of point 4
as follows.  After having completely determined the intersection point as outlined above (the predictor step), 
the procedure is repeated using the average values of the flow properties along each characteristic
(between points 1 and 4 and points 2 and 4) to recalculate the values found from Eqs. \ref{eqn:variables} 
and \ref{eqn:tpm} which are then used to find a new intersection point 4.  This procedure can be repeated until a 
specified tolerance level is reached.

\begin{equation}
	\begin{array}{c}
	y_-^{corr}= \frac{1}{2}(y_1 + y_4) \hspace{1.5cm} u_-^{corr} = \frac{1}{2}(u_1 + u_4)
	\hspace{1.5cm} v_-^{corr} = \frac{1}{2}(v_1 + v_4) \\
	y_+^{corr}= \frac{1}{2}(y_2 + y_4) \hspace{1.5cm} u_+^{corr} = \frac{1}{2}(u_2 + u_4)
	\hspace{1.5cm} v_+^{corr} = \frac{1}{2}(v_2 + v_4)
	\end{array}
\label{eqn:corrector}
\end{equation} 

